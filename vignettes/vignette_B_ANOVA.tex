\begin{lstlisting}[language=R, columns=fullflexible, basicstyle=\linespread{0.85}\small\ttfamily, stringstyle=\color{DarkGreen}, keywordstyle=\color{blue}, commentstyle=\color{DarkGreen},]
# Start the mixlm R package
library(mixlm, warn.conflicts = FALSE)
\end{lstlisting}

\subsection{Analysis of Variance
(ANOVA)}\label{analysis-of-variance-anova}

This example will focus on univariate ANOVA in various designs
including fixed and mixed effects, and briefly introduce multivariate
ANOVA. Models using random effects will be run through the mixlm
package.

\subsubsection{Simulated data}\label{simulated-data}

We will start by simulating some data to use in the examples below. In
this fictitious setup, milk yield is measured as a function of feed type
(low/high protein content), cow breed, bull identity, daughter and age.
The three first factors are crossed and balanced, while daughter is
nested under bull.

The yield is generated with a linear model with some noise.

\begin{lstlisting}[language=R, columns=fullflexible, basicstyle=\linespread{0.85}\small\ttfamily, stringstyle=\color{DarkGreen}, keywordstyle=\color{blue}, commentstyle=\color{DarkGreen},]
set.seed(123)
dat <- data.frame(
  feed  = factor(rep(rep(c("low","high"), each=6), 4)),
  breed = factor(rep(c("NRF","Hereford","Angus"), 16)),
  bull  = factor(rep(LETTERS[1:4], each = 12)),
  daughter = factor(rep(letters[1:4], 12)),
  age   = round(rnorm(48, mean = 36, sd = 5))
)
dat$yield <- 150*with(dat, 10 + 3 * as.numeric(feed) + as.numeric(breed) +
                        2 * as.numeric(bull) + 1 * as.numeric(sample(dat$daughter, 48)) +
                        0.5 * age + rnorm(48, sd = 2))
head(dat)
\end{lstlisting}
\begin{Verbatim}[fontsize=\small]
  feed    breed bull daughter age    yield
1  low      NRF    A        a  33 5541.301
2  low Hereford    A        b  35 5662.560
3  low    Angus    A        c  44 6354.182
4  low      NRF    A        d  36 6325.363
5  low Hereford    A        a  37 6144.458
6  low    Angus    A        b  45 6187.227
\end{Verbatim}

\subsubsection{Fixed effect models}\label{fixed-effect-models}

The simplest form of ANOVA is the fixed effect model. This model assumes
that the levels of the factor are fixed and that the only source of
variation is the factor itself.

\subsubsubsection{One-way ANOVA}\label{one-way-anova}

Here we assess only the feed effect on yield, i.e., the following model:
\[yield_{an} = \mu + feed_a + \epsilon_{an}\] where \(a\) is the feed
level and \(n\) is the observation within the feed level.

\begin{lstlisting}[language=R, columns=fullflexible, basicstyle=\linespread{0.85}\small\ttfamily, stringstyle=\color{DarkGreen}, keywordstyle=\color{blue}, commentstyle=\color{DarkGreen},]
mod <- lm(yield ~ feed, data = dat)
print(anova(mod))
\end{lstlisting}
\begin{Verbatim}[fontsize=\small]
Analysis of Variance Table

Response: yield
          Df   Sum Sq Mean Sq F value  Pr(>F)
feed       1   957882  957882  3.1072 0.08459 .
Residuals 46 14180730  308277
---
Signif. codes:  0 '***' 0.001 '**' 0.01 '*' 0.05 '.' 0.1 ' ' 1
\end{Verbatim}

In the ANOVA table one can look at Pr(\textgreater F) to see if the feed
factor has a significant effect on yield. The summary function can be
used to get more information about the underlying regression model.

\begin{lstlisting}[language=R, columns=fullflexible, basicstyle=\linespread{0.85}\small\ttfamily, stringstyle=\color{DarkGreen}, keywordstyle=\color{blue}, commentstyle=\color{DarkGreen},]
summary(mod)
\end{lstlisting}
\begin{Verbatim}[fontsize=\small]

Call:
lm(formula = yield ~ feed, data = dat)

Residuals:
    Min      1Q  Median      3Q     Max
-954.25 -480.25   75.11  470.14 1240.91

Coefficients:
            Estimate Std. Error t value Pr(>|t|)
(Intercept)  6354.28      80.14  79.290   <2e-16 ***
feed(low)    -141.27      80.14  -1.763   0.0846 .
---
Signif. codes:  0 '***' 0.001 '**' 0.01 '*' 0.05 '.' 0.1 ' ' 1

s: 555.2 on 46 degrees of freedom
Multiple R-squared: 0.06327,
Adjusted R-squared: 0.04291
F-statistic: 3.107 on 1 and 46 DF,  p-value: 0.08459
\end{Verbatim}

Basic model assessment can be done using the plot function.

\begin{lstlisting}[language=R, columns=fullflexible, basicstyle=\linespread{0.85}\small\ttfamily, stringstyle=\color{DarkGreen}, keywordstyle=\color{blue}, commentstyle=\color{DarkGreen},]
old.par <- par(mfrow=c(2,1), mar=c(4,4,2,0.5))
plot(mod, which = 1:2, ask=FALSE)
\end{lstlisting}

\includegraphics{Figures/FigCh8_model_assessment}

\begin{lstlisting}[language=R, columns=fullflexible, basicstyle=\linespread{0.85}\small\ttfamily, stringstyle=\color{DarkGreen}, keywordstyle=\color{blue}, commentstyle=\color{DarkGreen},]
par(old.par)
\end{lstlisting}

\subsubsubsection{Two-way crossed effects
ANOVA}\label{two-way-crossed-effects-anova}

Here we assess the feed and breed effects and their interaction effect
on yield, i.e., the following model:
\[yield_{abn} = \mu + feed_a + breed_b + (feed:breed)_{ab} + \epsilon_{abn}\]
where \(a\) is the feed level and \(n\) is the observation within the
feed level.

\begin{lstlisting}[language=R, columns=fullflexible, basicstyle=\linespread{0.85}\small\ttfamily, stringstyle=\color{DarkGreen}, keywordstyle=\color{blue}, commentstyle=\color{DarkGreen},]
mod <- lm(yield ~ feed*breed, data = dat)
print(anova(mod))
\end{lstlisting}
\begin{Verbatim}[fontsize=\small]
Analysis of Variance Table

Response: yield
           Df   Sum Sq Mean Sq F value Pr(>F)
feed        1   957882  957882  2.9904 0.0911 .
breed       2   284231  142116  0.4437 0.6447
feed:breed  2   443052  221526  0.6916 0.5064
Residuals  42 13453447  320320
---
Signif. codes:  0 '***' 0.001 '**' 0.01 '*' 0.05 '.' 0.1 ' ' 1
\end{Verbatim}

If the interaction effect is not significant, we can simplify the model
to: \[yield_{abn} = \mu + feed_a + breed_b + \epsilon_{abn}\]

\begin{lstlisting}[language=R, columns=fullflexible, basicstyle=\linespread{0.85}\small\ttfamily, stringstyle=\color{DarkGreen}, keywordstyle=\color{blue}, commentstyle=\color{DarkGreen},]
mod <- lm(yield ~ feed+breed, data = dat)
print(anova(mod))
\end{lstlisting}
\begin{Verbatim}[fontsize=\small]
Analysis of Variance Table

Response: yield
          Df   Sum Sq Mean Sq F value  Pr(>F)
feed       1   957882  957882  3.0329 0.08858 .
breed      2   284231  142116  0.4500 0.64055
Residuals 44 13896499  315830
---
Signif. codes:  0 '***' 0.001 '**' 0.01 '*' 0.05 '.' 0.1 ' ' 1
\end{Verbatim}

\subsubsubsection{Types of sums of
squares}\label{types-of-sums-of-squares}

The classical way of defining sums of squares are Type I, Type II, and
Type III, as described in the documentation of the Anova() function in
the car package.

\begin{lstlisting}[language=R, columns=fullflexible, basicstyle=\linespread{0.85}\small\ttfamily, stringstyle=\color{DarkGreen}, keywordstyle=\color{blue}, commentstyle=\color{DarkGreen},]
# Type I - Sequential testing, including one and one effect
print(anova(mod))
\end{lstlisting}
\begin{Verbatim}[fontsize=\small]
Analysis of Variance Table

Response: yield
          Df   Sum Sq Mean Sq F value  Pr(>F)
feed       1   957882  957882  3.0329 0.08858 .
breed      2   284231  142116  0.4500 0.64055
Residuals 44 13896499  315830
---
Signif. codes:  0 '***' 0.001 '**' 0.01 '*' 0.05 '.' 0.1 ' ' 1
\end{Verbatim}
\begin{lstlisting}[language=R, columns=fullflexible, basicstyle=\linespread{0.85}\small\ttfamily, stringstyle=\color{DarkGreen}, keywordstyle=\color{blue}, commentstyle=\color{DarkGreen},]

# Type II - Testing each term after all others,
# except ignoring the term's higher-order relatives
print(Anova(mod, type="II"))
\end{lstlisting}
\begin{Verbatim}[fontsize=\small]
Anova Table (Type II tests)

Response: yield
            Sum Sq Df F value  Pr(>F)
feed        957882  1  3.0329 0.08858 .
breed       284231  2  0.4500 0.64055
Residuals 13896499 44
---
Signif. codes:  0 '***' 0.001 '**' 0.01 '*' 0.05 '.' 0.1 ' ' 1
\end{Verbatim}
\begin{lstlisting}[language=R, columns=fullflexible, basicstyle=\linespread{0.85}\small\ttfamily, stringstyle=\color{DarkGreen}, keywordstyle=\color{blue}, commentstyle=\color{DarkGreen},]

# Type III - Testing each term after all others,
# including the term's higher-order relatives
print(Anova(mod, type="III"))
\end{lstlisting}
\begin{Verbatim}[fontsize=\small]
Anova Table (Type III tests)

Response: yield
                Sum Sq Df   F value  Pr(>F)
(Intercept) 1938092247  1 6136.5139 < 2e-16 ***
feed            957882  1    3.0329 0.08858 .
breed           284231  2    0.4500 0.64055
Residuals     13896499 44
---
Signif. codes:  0 '***' 0.001 '**' 0.01 '*' 0.05 '.' 0.1 ' ' 1
\end{Verbatim}

For the two-way ANOVA model, the Type I and Type II sums of squares are
the same, while Type III differs. With balanced data, this only happens
when the contrast coding is of the treatment/reference type.

\subsubsubsection{Contrast codings}\label{contrast-codings}

The contrast coding can be specified for each factor in the model. The
default is treatment coding, but other codings can be specified using
the \texttt{contrasts} Since we are running lm() through the mixlm
package, we can use the \texttt{contrasts} argument to specify the
coding for all effects simultaneously.

\begin{lstlisting}[language=R, columns=fullflexible, basicstyle=\linespread{0.85}\small\ttfamily, stringstyle=\color{DarkGreen}, keywordstyle=\color{blue}, commentstyle=\color{DarkGreen},]
# Sum coding, i.e., the sum of all levels is zero and all effects
# are orthogonal in the balanced case.
mod <- lm(yield ~ feed*breed, data = dat, contrasts="contr.sum")
print(Anova(mod, type="III"))
\end{lstlisting}
\begin{Verbatim}[fontsize=\small]
Anova Table (Type III tests)

Response: yield
                Sum Sq Df   F value Pr(>F)
(Intercept) 1938092247  1 6050.4845 <2e-16 ***
feed            957882  1    2.9904 0.0911 .
breed           284231  2    0.4437 0.6447
feed:breed      443052  2    0.6916 0.5064
Residuals     13453447 42
---
Signif. codes:  0 '***' 0.001 '**' 0.01 '*' 0.05 '.' 0.1 ' ' 1
\end{Verbatim}
\begin{lstlisting}[language=R, columns=fullflexible, basicstyle=\linespread{0.85}\small\ttfamily, stringstyle=\color{DarkGreen}, keywordstyle=\color{blue}, commentstyle=\color{DarkGreen},]

# Weighted coding, i.e., the sum of all levels is zero and the effects
# are weighted by the number of levels, effect-wise.
mod <- lm(yield ~ feed*breed, data = dat, contrasts="contr.weighted")
print(Anova(mod, type="III"))
\end{lstlisting}
\begin{Verbatim}[fontsize=\small]
Anova Table (Type III tests)

Response: yield
                Sum Sq Df   F value Pr(>F)
(Intercept) 1938092247  1 6050.4845 <2e-16 ***
feed            957882  1    2.9904 0.0911 .
breed           284231  2    0.4437 0.6447
feed:breed      443052  2    0.6916 0.5064
Residuals     13453447 42
---
Signif. codes:  0 '***' 0.001 '**' 0.01 '*' 0.05 '.' 0.1 ' ' 1
\end{Verbatim}

Instead of specifying the contrasts in a specific model, it is also
possible to set the contrasts globally for the session. This means that
all subsequent models, unless specified otherwise, will use the
specified contrasts.

\begin{lstlisting}[language=R, columns=fullflexible, basicstyle=\linespread{0.85}\small\ttfamily, stringstyle=\color{DarkGreen}, keywordstyle=\color{blue}, commentstyle=\color{DarkGreen},]
options(contrasts = c("contr.sum", "contr.poly"))
\end{lstlisting}

\subsubsubsection{Covariates in ANOVA}\label{covariates-in-anova}

Adding covariates to an ANOVA model is straightforward. Here we add the
age of the cow as a covariate to the two-way ANOVA model. The model
becomes:
\[yield_{abn} = \mu + feed_a + breed_b + (feed:breed)_{ab} + age\cdot x_{abn} + \epsilon_{abn},\]
where \(x_{abn}\) is the age of the cow and \(age\) is its linear
coefficient.

\begin{lstlisting}[language=R, columns=fullflexible, basicstyle=\linespread{0.85}\small\ttfamily, stringstyle=\color{DarkGreen}, keywordstyle=\color{blue}, commentstyle=\color{DarkGreen},]
mod <- lm(yield ~ feed*breed + age, data = dat)
print(Anova(mod, type="II"))
\end{lstlisting}
\begin{Verbatim}[fontsize=\small]
Anova Table (Type II tests)

Response: yield
            Sum Sq Df F value    Pr(>F)
feed        938176  1  4.3325   0.04368 *
breed       368528  2  0.8509   0.43442
age        4575110  1 21.1278 4.064e-05 ***
feed:breed   73643  2  0.1700   0.84422
Residuals  8878337 41
---
Signif. codes:  0 '***' 0.001 '**' 0.01 '*' 0.05 '.' 0.1 ' ' 1
\end{Verbatim}

\subsubsubsection{Fixed effect nested
ANOVA}\label{fixed-effect-nested-anova}

In the case of nested factors, we can specify this in the model. In the
current model we assume that bulls are fixed effects that we are
interested in and that daughters are nested under bulls. In this case
the daughters do not have any special attributes that would interefere
with the estimation of the bull effect, so we do not have to assume that
they are random effects. The model becomes:
\[yield_{abn} = \mu + bull_a + daugter_{b(a)} + \epsilon_{abn},\]

\begin{lstlisting}[language=R, columns=fullflexible, basicstyle=\linespread{0.85}\small\ttfamily, stringstyle=\color{DarkGreen}, keywordstyle=\color{blue}, commentstyle=\color{DarkGreen},]
mod <- lm(yield ~ bull + daughter%in%bull, data = dat)
print(Anova(mod, type="II"))
\end{lstlisting}
\begin{Verbatim}[fontsize=\small]
Anova Table (Type II tests)

Response: yield
               Sum Sq Df F value    Pr(>F)
bull          6293005  3  8.9924 0.0001822 ***
bull:daughter 1380934 12  0.4933 0.9034247
Residuals     7464672 32
---
Signif. codes:  0 '***' 0.001 '**' 0.01 '*' 0.05 '.' 0.1 ' ' 1
\end{Verbatim}

\subsubsection{Linear mixed models}\label{linear-mixed-models}

Adding random effects to a model can be done either using least squares
modelling through the mixlm package or using ML/REML estimation through
the lme4 package (or similar).

\subsubsubsection{Classical - mixlm}\label{classical---mixlm}

Using the mixlm package, we specify the random effects using the
\texttt{r()} function. If we assume that the bull is a random selection
from the population of bulls, we can specify this as a random effect
when focusing on feed. The model looks like a fixed effect model, but
the error structure is different:
\[yield_{abn} = \mu + feed_a + breed_b + (feed:breed)_{ab} + \epsilon_{abn}\]

\begin{lstlisting}[language=R, columns=fullflexible, basicstyle=\linespread{0.85}\small\ttfamily, stringstyle=\color{DarkGreen}, keywordstyle=\color{blue}, commentstyle=\color{DarkGreen},]
mod <- lm(yield ~ feed*r(bull), data = dat)
print(Anova(mod, type="II"))
\end{lstlisting}
\begin{Verbatim}[fontsize=\small]
Analysis of variance (unrestricted model)
Response: yield
             Mean Sq     Sum Sq Df F value Pr(>F)
feed       957881.78  957881.78  1    3.75 0.1481
bull      2097668.39 6293005.18  3    8.22 0.0587
feed:bull  255297.76  765893.29  3    1.43 0.2472
Residuals  178045.79 7121831.56 40       -      -

            Err.term(s) Err.df VC(SS)
1 feed              (3)      3  fixed
2 bull              (3)      3 153531
3 feed:bull         (4)     40  12875
4 Residuals           -      - 178046
(VC = variance component)

          Expected mean squares
feed      (4) + 6 (3) + 24 Q[1]
bull      (4) + 6 (3) + 12 (2)
feed:bull (4) + 6 (3)
Residuals (4)
\end{Verbatim}

In addition to the ordinary ANOVA table, an overview of variance
components, and expected mean squares are printed.

\subsubsubsection{Restrictions}\label{restrictions}

The mixlm package has unrestricted models as default, but it is possible
to turn on restriction.

\begin{lstlisting}[language=R, columns=fullflexible, basicstyle=\linespread{0.85}\small\ttfamily, stringstyle=\color{DarkGreen}, keywordstyle=\color{blue}, commentstyle=\color{DarkGreen},]
mod <- lm(yield ~ feed*r(bull), data = dat, unrestricted = FALSE)
print(Anova(mod, type="II"))
\end{lstlisting}
\begin{Verbatim}[fontsize=\small]
Analysis of variance (restricted model)
Response: yield
             Mean Sq     Sum Sq Df F value Pr(>F)
feed       957881.78  957881.78  1    3.75 0.1481
bull      2097668.39 6293005.18  3   11.78 0.0000
feed:bull  255297.76  765893.29  3    1.43 0.2472
Residuals  178045.79 7121831.56 40       -      -

            Err.term(s) Err.df VC(SS)
1 feed              (3)      3  fixed
2 bull              (4)     40 159969
3 feed:bull         (4)     40  12875
4 Residuals           -      - 178046
(VC = variance component)

          Expected mean squares
feed      (4) + 6 (3) + 24 Q[1]
bull      (4) + 12 (2)
feed:bull (4) + 6 (3)
Residuals (4)
\end{Verbatim}

This effects which tests are performed and how the variance components
are estimated.

\subsubsubsection{Repeated Measures}\label{repeated-measures}

A repeated measures model can be a mixed model with a random effect for
the repeated measures, where the repeated measures are nested under the
subjects. Longitudinal data is a common example of repeated measures
data, where the replicates are repetitions over time within subject. If
we subset the simulated data, we can add a longitudinal effect to the
model, in this case a random variation over three time-points.

\begin{lstlisting}[language=R, columns=fullflexible, basicstyle=\linespread{0.85}\small\ttfamily, stringstyle=\color{DarkGreen}, keywordstyle=\color{blue}, commentstyle=\color{DarkGreen},]
long <- dat[c(1:4,9:12), c("feed", "daughter", "yield")]
long <- rbind(long, long, long)
long$time  <- factor(rep(1:3, each=8))
long$yield <- long$yield + rnorm(24, sd = 100)
plot(yield~daughter, data=long)
\end{lstlisting}

\includegraphics{Figures/FigCh8_repeated_measures}

Now we have a feed effect, individuals (daughters), and a time effect
repeated inside daughters, with the model:
\[yield_{ait} = \mu + daughter_i + feed_a + time_t + (feed:time)_{at} + \epsilon_{ait}\]

\begin{lstlisting}[language=R, columns=fullflexible, basicstyle=\linespread{0.85}\small\ttfamily, stringstyle=\color{DarkGreen}, keywordstyle=\color{blue}, commentstyle=\color{DarkGreen},]
mod <- lm(yield ~ r(daughter) + feed*r(time), data = long, unrestricted=FALSE)
print(Anova(mod, type="II"))
\end{lstlisting}
\begin{Verbatim}[fontsize=\small]
Analysis of variance (restricted model)
Response: yield
            Mean Sq     Sum Sq Df F value Pr(>F)
daughter  860788.48 2582365.45  3   10.57 0.0005
feed      294289.41  294289.41  1  102.53 0.0096
time       26950.85   53901.69  2    0.33 0.7234
feed:time   2870.28    5740.56  2    0.04 0.9655
Residuals  81462.41 1221936.15 15       -      -

            Err.term(s) Err.df VC(SS)
1 daughter          (5)     15 129888
2 feed              (4)      2  fixed
3 time              (5)     15  -6814
4 feed:time         (5)     15 -19648
5 Residuals           -      -  81462
(VC = variance component)

          Expected mean squares
daughter  (5) + 6 (1)
feed      (5) + 4 (4) + 12 Q[2]
time      (5) + 8 (3)
feed:time (5) + 4 (4)
Residuals (5)
\end{Verbatim}

\subsubsubsection{REML}\label{reml}

REML estiamation can be done directly with the lme4 package, but we can
also do this through the mixlm package, leveraging the r() function.

\begin{lstlisting}[language=R, columns=fullflexible, basicstyle=\linespread{0.85}\small\ttfamily, stringstyle=\color{DarkGreen}, keywordstyle=\color{blue}, commentstyle=\color{DarkGreen},]
mod <- lm(yield ~ feed*r(bull), data = dat, REML = TRUE)
print(Anova(mod, type="III"))
\end{lstlisting}
\begin{Verbatim}[fontsize=\small]
Analysis of Deviance Table (Type III Wald chisquare tests)

Response: yield
              Chisq Df Pr(>Chisq)
(Intercept) 923.918  1    < 2e-16 ***
feed          3.752  1    0.05274 .
---
Signif. codes:  0 '***' 0.001 '**' 0.01 '*' 0.05 '.' 0.1 ' ' 1
\end{Verbatim}

To see how mixlm transforms the model to lme4, we can print the model.

\begin{lstlisting}[language=R, columns=fullflexible, basicstyle=\linespread{0.85}\small\ttfamily, stringstyle=\color{DarkGreen}, keywordstyle=\color{blue}, commentstyle=\color{DarkGreen},]
print(mod)
\end{lstlisting}
\begin{Verbatim}[fontsize=\small]
Linear mixed model fit by REML ['lmerMod']
Formula: yield ~ feed + (1 | bull) + (1 | feed:bull)
   Data: dat
REML criterion at convergence: 702.8962
Random effects:
 Groups    Name        Std.Dev.
 feed:bull (Intercept) 113.5
 bull      (Intercept) 391.8
 Residual              422.0
Number of obs: 48, groups:  feed:bull, 8; bull, 4
Fixed Effects:
(Intercept)        feed1
     6354.3       -141.3
\end{Verbatim}

We observe that (1 \textbar{} bull) and (1 \textbar{} \url{feed:bull})
are added to the model, which means that random intercepts are added for
both bull and the interaction.

\subsubsection{Multivariate ANOVA
(MANOVA)}\label{multivariate-anova-manova}

Basic multivariate ANOVA can be done using the lm() function, if we
create a matrix of responses. In this case we add a mastitis effect to
the model.

\begin{lstlisting}[language=R, columns=fullflexible, basicstyle=\linespread{0.85}\small\ttfamily, stringstyle=\color{DarkGreen}, keywordstyle=\color{blue}, commentstyle=\color{DarkGreen},]
dat$mastitis <- as.numeric(dat$breed) + as.numeric(dat$feed) + rnorm(48, sd = 1)
\end{lstlisting}

The model becomes:
\[[yield_{abn} | mastitis_{abn}] = \mu + feed_a + breed_b + (feed:breed)_{ab} + \epsilon_{abn},\]
where each of the model terms now are vectors matching the number of
responses.

\begin{lstlisting}[language=R, columns=fullflexible, basicstyle=\linespread{0.85}\small\ttfamily, stringstyle=\color{DarkGreen}, keywordstyle=\color{blue}, commentstyle=\color{DarkGreen},]
mod <- lm(cbind(yield,mastitis) ~ feed*breed, data = dat)
print(Anova(mod, type="II"))
\end{lstlisting}
\begin{Verbatim}[fontsize=\small]

Type II MANOVA Tests: Pillai test statistic
           Df test stat approx F num Df den Df    Pr(>F)
feed        1   0.35268  11.1691      2     41 0.0001343 ***
breed       2   0.52220   7.4207      4     84 3.618e-05 ***
feed:breed  2   0.04003   0.4289      4     84 0.7874006
---
Signif. codes:  0 '***' 0.001 '**' 0.01 '*' 0.05 '.' 0.1 ' ' 1
\end{Verbatim}

The test statistics are joint for all responses, here in the form of the
defaul Pillai's test statistics. Other statistics can be produced as
follows:

\begin{lstlisting}[language=R, columns=fullflexible, basicstyle=\linespread{0.85}\small\ttfamily, stringstyle=\color{DarkGreen}, keywordstyle=\color{blue}, commentstyle=\color{DarkGreen},]
print(Anova(mod, type="II", test="Wilks"))
\end{lstlisting}
\begin{Verbatim}[fontsize=\small]

Type II MANOVA Tests: Wilks test statistic
           Df test stat approx F num Df den Df    Pr(>F)
feed        1   0.64732  11.1691      2     41 0.0001343 ***
breed       2   0.47861   9.1321      4     82 3.752e-06 ***
feed:breed  2   0.96019   0.4206      4     82 0.7933107
---
Signif. codes:  0 '***' 0.001 '**' 0.01 '*' 0.05 '.' 0.1 ' ' 1
print(Anova(mod, type="II", test="Hotelling-Lawley"))

Type II MANOVA Tests: Hotelling-Lawley test statistic
           Df test stat approx F num Df den Df    Pr(>F)
feed        1   0.54483  11.1691      2     41 0.0001343 ***
breed       2   1.08769  10.8769      4     80 4.311e-07 ***
feed:breed  2   0.04123   0.4123      4     80 0.7992719
---
Signif. codes:  0 '***' 0.001 '**' 0.01 '*' 0.05 '.' 0.1 ' ' 1
\end{Verbatim}
\begin{lstlisting}[language=R, columns=fullflexible, basicstyle=\linespread{0.85}\small\ttfamily, stringstyle=\color{DarkGreen}, keywordstyle=\color{blue}, commentstyle=\color{DarkGreen},]
print(Anova(mod, type="II", test="Roy"))
\end{lstlisting}
\begin{Verbatim}[fontsize=\small]

Type II MANOVA Tests: Roy test statistic
           Df test stat approx F num Df den Df    Pr(>F)
feed        1   0.54483  11.1691      2     41 0.0001343 ***
breed       2   1.08612  22.8086      2     42 1.967e-07 ***
feed:breed  2   0.03470   0.7288      2     42 0.4885053
---
Signif. codes:  0 '***' 0.001 '**' 0.01 '*' 0.05 '.' 0.1 ' ' 1
\end{Verbatim}

The summary function can be used to get more information about the
underlying regression model, here revealing the regressions are
performed separately for each response.

\begin{lstlisting}[language=R, columns=fullflexible, basicstyle=\linespread{0.85}\small\ttfamily, stringstyle=\color{DarkGreen}, keywordstyle=\color{blue}, commentstyle=\color{DarkGreen},]
summary(mod)
\end{lstlisting}
\begin{Verbatim}[fontsize=\small]
Response yield :

Call:
lm(formula = yield ~ feed * breed, data = dat)

Residuals:
     Min       1Q   Median       3Q      Max
-1148.50  -489.54   -25.54   349.48  1142.16

Coefficients:
                           Estimate Std. Error t value Pr(>|t|)
(Intercept)                 6354.28      81.69  77.785   <2e-16 ***
feed(high)                  -141.27      81.69  -1.729   0.0911 .
breed(Angus)                 -71.72     115.53  -0.621   0.5381
breed(Hereford)              -35.02     115.53  -0.303   0.7633
feed(high):breed(Angus)      -46.25     115.53  -0.400   0.6909
feed(high):breed(Hereford)   133.76     115.53   1.158   0.2535
---
Signif. codes:  0 '***' 0.001 '**' 0.01 '*' 0.05 '.' 0.1 ' ' 1

Residual standard error: 566 on 42 degrees of freedom
Multiple R-squared:  0.1113, Adjusted R-squared:  0.00552
F-statistic: 1.052 on 5 and 42 DF,  p-value: 0.4003


Response mastitis :

Call:
lm(formula = mastitis ~ feed * breed, data = dat)

Residuals:
    Min      1Q  Median      3Q     Max
-2.4490 -0.6960 -0.1311  0.6907  2.1462

Coefficients:
                             Estimate Std. Error t value Pr(>|t|)
(Intercept)                 3.4811963  0.1603909  21.704  < 2e-16 ***
feed(high)                 -0.7450433  0.1603909  -4.645 3.34e-05 ***
breed(Angus)               -1.2830903  0.2268270  -5.657 1.24e-06 ***
breed(Hereford)            -0.0833432  0.2268270  -0.367    0.715
feed(high):breed(Angus)    -0.1066673  0.2268270  -0.470    0.641
feed(high):breed(Hereford)  0.0008313  0.2268270   0.004    0.997
---
Signif. codes:  0 '***' 0.001 '**' 0.01 '*' 0.05 '.' 0.1 ' ' 1

Residual standard error: 1.111 on 42 degrees of freedom
Multiple R-squared:  0.6164, Adjusted R-squared:  0.5707
F-statistic:  13.5 on 5 and 42 DF,  p-value: 7.279e-08
\end{Verbatim}
