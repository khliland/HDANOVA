\texttt{\{r,\ include=FALSE\}\ knitr::opts\_chunk\$set(\ \ \ collapse\ =\ TRUE,\ \ \ comment\ =\ "\#\textgreater{}",\ \ \ fig.width=6,\ \ \ \ fig.height=4\ )\ \#\ Legge\ denne\ i\ YAML\ på\ toppen\ for\ å\ skrive\ ut\ til\ tex\ \#output:\ \ \#\ \ pdf\_document:\ \ \#\ \ \ \ keep\_tex:\ true\ \#\ Original:\ \#\ \ rmarkdown::html\_vignette:\ \#\ \ \ \ toc:\ true}

\texttt{\{r\ setup\}\ \#\ Start\ the\ HDANOVA\ R\ package\ library(HDANOVA)}

\section{Analysis of Variance Simultaneous Component Analysis
(ASCA)}\label{analysis-of-variance-simultaneous-component-analysis-asca}

The ANOVA part of ASCA includes all the possible variations of ANOVA
demonstrated in the ANOVA section and more. Also, generalized linear
models (GLM) can be used. The following theory will be exemplified:

\begin{itemize}
\tightlist
\item
  Basic ASCA

  \begin{itemize}
  \tightlist
  \item
    Permutation testing
  \item
    Random effects
  \item
    Scores and loadings
  \item
    Data and confidence ellipsoids
  \item
    Combined effects
  \item
    Numeric effects
  \end{itemize}
\item
  ANOVA-PCA (APCA)
\item
  PC-ANOVA
\item
  MSCA
\item
  LiMM-PCA
\end{itemize}

\subsection{Basic ASCA}\label{basic-asca}

The ANOVA part of ASCA includes all the possible variations of ANOVA
demonstrated in the ANOVA vignette and more. Also generalized linear
models (GLM) can be used. We start by demonstrating ASCA with a fixed
effect model of two factors with interactions and ordinary PCA on the
effect matrices.

```\{r\} \# Load Candy data data(candies)

\section{Fit ASCA model}\label{fit-asca-model}

mod \textless- asca(assessment \textasciitilde{} candy*assessor,
data=candies) summary(mod)

\begin{verbatim}

The summary shows that the candy effect is the largest by far. 

### Permutation testing
To get more insight, we can perform permutation testing of the factors. Here we use approximate permutation.

```{r}
# Permutation testing (default = 1000 permutations, if not specified)
mod <- asca(assessment ~ candy*assessor, data=candies, permute=TRUE)
summary(mod)
\end{verbatim}

Here we see that all effects are significant, where the Candy effect is
the dominating one. This can also be visualised by looking at the
sums-of-squares values obtained under permutation compared to the
original value.

\texttt{\{r\}\ permutationplot(mod,\ factor\ =\ "assessor")}

\subsubsection{Random effects}\label{random-effects}

One can argue that the assessors are random effects, thus should be
handled as such in the model. We can do this by adding r() around the
assessor term. See also LiMM-PCA below for the REML estimation version.

\texttt{\{r\}\ \#\ Fit\ ASCA\ model\ with\ random\ assessor\ mod.mixed\ \textless{}-\ asca(assessment\ \textasciitilde{}\ candy*r(assessor),\ data=candies,\ permute=TRUE)\ summary(mod.mixed)}

\subsubsection{Scores and loadings}\label{scores-and-loadings}

The effects can be visualised through, e.g., loading and score plots to
assess the relations between variables, objects and factors. If a factor
is not specified, the first factor is plotted.

\texttt{\{r,\ fig.width=4.5,\ fig.height=7\}\ par.old\ \textless{}-\ par(mfrow=c(2,1),\ mar=c(4,4,2,1))\ loadingplot(mod,\ scatter=TRUE,\ labels="names",\ main="Candy\ loadings")\ scoreplot(mod,\ main="Candy\ scores")\ par(par.old)}

A specific factor can be plotted by specifying the factor name or number

\texttt{\{r,\ fig.width=4.5,\ fig.height=7\}\ par.old\ \textless{}-\ par(mfrow=c(2,1),\ mar=c(4,4,2,1))\ loadingplot(mod,\ factor="assessor",\ scatter=TRUE,\ labels="names",\ main="Assessor\ loadings")\ scoreplot(mod,\ factor="assessor",\ main="Assessor\ scores")\ par(par.old)}

Score plots can be modified, e.g., omitting backprojections or adding
spider plots.

\texttt{\{r,\ fig.width=4.5,\ fig.height=7\}\ par.old\ \textless{}-\ par(mfrow=c(2,1),\ mar=c(4,4,2,1))\ scoreplot(mod,\ factor="assessor",\ main="Assessor\ scores",\ projections=FALSE)\ scoreplot(mod,\ factor="assessor",\ main="Assessor\ scores",\ spider=TRUE)\ par(par.old)}

And scores and loadings can be extracted for further analysis.

\texttt{\{r\}\ L\ \textless{}-\ loadings(mod,\ factor="candy")\ head(L)\ S\ \textless{}-\ scores(mod,\ factor="candy")\ head(S)}

\subsubsection{Data ellipsoids and confidence
ellipsoids}\label{data-ellipsoids-and-confidence-ellipsoids}

To emphasize factor levels or assess factor level differences, we can
add data ellipsoids or confidence ellipsoids to the score plot. The
confidence ellipsoids are built on the assumption of balanced data, and
their theories are built around crossed designs.

\texttt{\{r,\ fig.width=4.5,\ fig.height=7\}\ par.old\ \textless{}-\ par(mfrow=c(2,1),\ mar=c(4,4,2,1))\ scoreplot(mod,\ ellipsoids="data",\ factor="candy",\ main="Data\ ellipsoids")\ scoreplot(mod,\ ellipsoids="confidence",\ factor="candy",\ main="Confidence\ ellipsoids")\ par(par.old)}

If we repeat this for the mixed model, we see that both types of
ellipsoids change together with the change in denominator term in the
underlying ANOVA model. It should be noted that the theory for
confidence ellipsoids in mixed models is not fully developed, so
interpretation should be done with caution.

\texttt{\{r,\ fig.width=4.5,\ fig.height=7\}\ par.old\ \textless{}-\ par(mfrow=c(2,1),\ mar=c(4,4,2,1))\ scoreplot(mod.mixed,\ ellipsoids="data",\ factor="candy",\ main="Data\ ellipsoids")\ scoreplot(mod.mixed,\ ellipsoids="confidence",\ factor="candy",\ main="Confidence\ ellipsoids")\ par(par.old)}

\subsubsection{Combined effects}\label{combined-effects}

In some cases, it can be of interest to combine effects in ASCA. Here,
we use an example with the Caldana data where we combine the
\emph{light} effect with the \emph{time:light} interaction using the
\emph{comb()} function.

```\{r\} \# Load Caldana data data(caldana)

\section{Combined effects}\label{combined-effects-1}

mod.comb \textless- asca(compounds \textasciitilde{} time + comb(light +
light:time), data=caldana) summary(mod.comb)

\section{Scores plotted as a function of
time}\label{scores-plotted-as-a-function-of-time}

par.old \textless- par(mfrow=c(2,1), mar = c(4,4,1,1))
timeplot(mod.comb, factor=``light'', time=``time'', comb=2, comp=1,
x\_time=TRUE) timeplot(mod.comb, factor=``light'', time=``time'',
comb=2, comp=2, x\_time=TRUE) par(par.old)

\begin{verbatim}

### Quantitative effects
Quantitative effects, so-called covariates, can also be included in a model, though their use
in ASCA are limited to ANOVA estimation and explained variance, not being used in
subsequent PCA or permutation testing. We demonstrate this using the Caldana data
again, but now recode the time effect to a quantitative effect, meaning it will be
handled as a linear continuous effect.

```{r}
caldanaNum <- caldana
caldanaNum$time <- as.numeric(as.character(caldanaNum$time))
mod.num <- asca(compounds ~ time*light, data = caldanaNum)
summary(mod.num)
\end{verbatim}

\subsection{ANOVA-PCA (APCA)}\label{anova-pca-apca}

APCA differs from ASCA by adding the error term to the model before
performing PCA instead of backprojecting errors afterwards.

\texttt{\{r\}\ \#\ Fit\ APCA\ model\ modp\ \textless{}-\ apca(assessment\ \textasciitilde{}\ candy*assessor,\ data=candies)\ summary(modp)}

Plot scores and loadings.
\texttt{\{r,\ fig.width=4.5,\ fig.height=7\}\ par.old\ \textless{}-\ par(mfrow=c(2,1),\ mar=c(4,4,2,1))\ loadingplot(modp,\ scatter=TRUE,\ labels="names",\ main="Candy\ loadings")\ scoreplot(modp,\ main="Candy\ scores")\ par(par.old)}

\subsection{PC-ANOVA}\label{pc-anova}

In PC-ANOVA, a PCA is first applied to the data before the scores are
subjected to ANOVA, effectively reversing the roles in ASCA. This means
there will be one or more ANOVA tables in the summary of the output. In
this example, we have chosen to use the number of components that
explain at least 90\% of the variation of the data.
\texttt{\{r\}\ mod.pc\ \textless{}-\ pcanova(assessment\ \textasciitilde{}\ candy\ *\ assessor,\ data\ =\ candies,\ ncomp\ =\ 0.9)\ print(mod.pc)\ summary(mod.pc)}

When creating score and loading plots for PC-ANOVA, the `global' scores
and loadings will be shown, but the factors can still be used for
manipulating the symbols.
\texttt{\{r,\ fig.width=4.5,\ fig.height=7\}\ par.old\ \textless{}-\ par(mfrow=c(2,1),\ mar=c(4,4,2,1))\ loadingplot(mod.pc,\ scatter=TRUE,\ labels="names",\ main="Global\ loadings")\ scoreplot(mod.pc,\ main="Global\ scores")\ par(par.old)}

\subsection{MSCA}\label{msca}

Multilevel Simultaneous Component Analysis (MSCA) is a version of ASCA
that assumes a single factor to be used as a between-individuals factor,
while the the within-individuals is assumed implicitly.

```\{r\} \# Default MSCA model with a single factor mod.msca \textless-
msca(assessment \textasciitilde{} candy, data=candies) summary(mod.msca)

\section{Scoreplot for between-individuals
factor}\label{scoreplot-for-between-individuals-factor}

scoreplot(mod.msca)

\section{Scoreplot of within-individuals
factor}\label{scoreplot-of-within-individuals-factor}

scoreplot(mod.msca, factor=``within'')

\section{.. per factor level}\label{per-factor-level}

par.old \textless- par(mfrow=c(3,2), mar=c(4,4,2,1), mgp=c(2,0.7,0))
for(i in 1:length(mod.msca\(scores.within))
 scoreplot(mod.msca, factor="within", within_level=i,
           main=paste0("Level: ", names(mod.msca\)scores.within){[}i{]}),
panel.first=abline(v=0,h=0,col=``gray'',lty=2)) par(par.old)

\begin{verbatim}



## LiMM-PCA

A version of LiMM-PCA is also implemented in HDANOVA. It combines REML estimated
mixed models with PCA and scales the backprojected errors according to degrees of freedom
or effective dimensions (user choice).

```{r}
# Default LiMM-PCA model with two factors and interaction, 8 PCA components
mod.reml <- limmpca(assessment ~ candy*r(assessor), data=candies, pca.in=8)
summary(mod.reml)
scoreplot(mod.reml, factor="candy")
\end{verbatim}

One can also use least squares estimation without REML. This affects the
random effects and scaling of backprojections.

\texttt{\{r\}\ \#\ LiMM-PCA\ with\ least\ squares\ estimation\ and\ 8\ PCA\ components\ mod.ls\ \textless{}-\ limmpca(assessment\ \textasciitilde{}\ candy*r(assessor),\ data=candies,\ REML=NULL,\ pca.in=8)\ summary(mod.ls)\ scoreplot(mod.ls)}
