\subsection{Basic ASCA}\label{basic-asca}

The ANOVA part of ASCA includes all the possible variations of ANOVA
demonstrated in the ANOVA vignette and more. Also generalized linear
models (GLM) can be used. We start by demonstrating ASCA with a fixed
effect model of two factors with interactions and ordinary PCA on the
effect matrices.

\begin{lstlisting}[language=R, columns=fullflexible, basicstyle=\linespread{0.85}\small\ttfamily, stringstyle=\color{DarkGreen}, keywordstyle=\color{blue}, commentstyle=\color{DarkGreen},]
# Load Candy data
data(candies)
\end{lstlisting}

\subsection{Fit ASCA model}\label{fit-asca-model}

mod <- asca(assessment ~ candy*assessor, data=candies)
summary(mod)
\end{lstlisting}

The summary shows that the candy effect is the largest by far.

\subsubsection{Permutation testing}
To get more insight, we can perform permutation testing of the factors.

\begin{lstlisting}[language=R, columns=fullflexible, basicstyle=\linespread{0.85}\small\ttfamily, stringstyle=\color{DarkGreen}, keywordstyle=\color{blue}, commentstyle=\color{DarkGreen},]
# Permutation testing (default = 1000 permutations, if not specified)
mod <- asca(assessment ~ candy*assessor, data=candies, permute=TRUE)
summary(mod)
\end{lstlisting}

Here we see that the candy effect is significant, while the assessor
effect and interaction are not. This can also be visualised by looking
at the sums-of-squares values obtained under permutation compared to the
original value.

\begin{lstlisting}[language=R, columns=fullflexible, basicstyle=\linespread{0.85}\small\ttfamily, stringstyle=\color{DarkGreen}, keywordstyle=\color{blue}, commentstyle=\color{DarkGreen},]
permutationplot(mod, factor = "assessor")}
\end{lstlisting}

\subsubsubsection{Random effects}\label{random-effects}

One can argue that the assessors are random effects, thus should be
handled as such in the model. We can do this by adding r() around the
assessor term.

\begin{lstlisting}[language=R, columns=fullflexible, basicstyle=\linespread{0.85}\small\ttfamily, stringstyle=\color{DarkGreen}, keywordstyle=\color{blue}, commentstyle=\color{DarkGreen},]
# Fit ASCA model with random assessor
mod.mixed <- asca(assessment ~ candy*r(assessor), data=candies, permute=TRUE)
summary(mod.mixed)
\end{lstlisting}

\subsubsubsection{Scores and loadings}\label{scores-and-loadings}

The effects can be visualised through, e.g., loading and score plots to
assess the relations between variables, objects and factors. If a factor
is not specified, the first factor is plotted.

\begin{lstlisting}[language=R, columns=fullflexible, basicstyle=\linespread{0.85}\small\ttfamily, stringstyle=\color{DarkGreen}, keywordstyle=\color{blue}, commentstyle=\color{DarkGreen},]
par.old <- par(mfrow=c(2,1), mar=c(4,4,2,1))
loadingplot(mod, scatter=TRUE, labels="names", main="Candy loadings")
scoreplot(mod, main="Candy scores")
par(par.old)
\end{lstlisting}

A specific factor can be plotted by specifying the factor name or number

\begin{lstlisting}[language=R, columns=fullflexible, basicstyle=\linespread{0.85}\small\ttfamily, stringstyle=\color{DarkGreen}, keywordstyle=\color{blue}, commentstyle=\color{DarkGreen},]
par.old <- par(mfrow=c(2,1), mar=c(4,4,2,1))
loadingplot(mod, factor="assessor", scatter=TRUE, labels="names", main="Assessor loadings")
scoreplot(mod, factor="assessor", main="Assessor scores")
par(par.old)
\end{lstlisting}

And scores and loadings can be extracted for further analysis.

\begin{lstlisting}[language=R, columns=fullflexible, basicstyle=\linespread{0.85}\small\ttfamily, stringstyle=\color{DarkGreen}, keywordstyle=\color{blue}, commentstyle=\color{DarkGreen},]
L <- loadings(mod, factor="candy")
head(L)
S <- scores(mod, factor="candy")
head(S)
\end{lstlisting}

\subsubsubsection{Data ellipsoids and confidence
ellipsoids}\label{data-ellipsoids-and-confidence-ellipsoids}

To emphasize factor levels or assess factor level differences, one can
add data ellipsoids or confidence ellipsoids to the score plot. The
confidence ellipsoids are built on the assumption of balanced data, and
their theory are are built around crossed designs.

\begin{lstlisting}[language=R, columns=fullflexible, basicstyle=\linespread{0.85}\small\ttfamily, stringstyle=\color{DarkGreen}, keywordstyle=\color{blue}, commentstyle=\color{DarkGreen},]
par.old <- par(mfrow=c(2,1), mar=c(4,4,2,1))
scoreplot(mod, ellipsoids="data", factor="candy", main="Data ellipsoids")
scoreplot(mod, ellipsoids="confidence", factor="candy", main="Confidence ellipsoids")
par(par.old)
\end{lstlisting}

If we repeat this for the mixed model, we see that the both types of
ellipsoids change together with the change in denominator term in the
underlying ANOVA model.

\begin{lstlisting}[language=R, columns=fullflexible, basicstyle=\linespread{0.85}\small\ttfamily, stringstyle=\color{DarkGreen}, keywordstyle=\color{blue}, commentstyle=\color{DarkGreen},]
par.old <- par(mfrow=c(2,1), mar=c(4,4,2,1))
scoreplot(mod.mixed, ellipsoids="data", factor="candy", main="Data ellipsoids")
scoreplot(mod.mixed, ellipsoids="confidence", factor="candy", main="Confidence ellipsoids")
par(par.old)
\end{lstlisting}

\subsubsubsection{Combined effects}\label{combined-effects}

In some cases it can be of interest to combine effects in ASCA. Here, we
use an example with the Caldana data where we combine the \emph{light}
effect with the \emph{time:light} interaction using the \emph{comb()}
function.

\begin{lstlisting}[language=R, columns=fullflexible, basicstyle=\linespread{0.85}\small\ttfamily, stringstyle=\color{DarkGreen}, keywordstyle=\color{blue}, commentstyle=\color{DarkGreen},]
# Load Caldana data
data(caldana)

mod.comb <- asca(compounds ~ time + comb(light + light:time), data=caldana)
summary(mod.comb)
\end{lstlisting}

\subsubsubsection{Numeric effects}\label{numeric-effects}
Numeric effects, so-called covariates, can also be included in a model, though their
in ASCA are limited to ANOVA estimation and explained variance, not being used in
subsequent PCA or permutation testing. We demonstrate this using the Caldana data
again, but now recode the time effect to a numeric effect, meaning it will be
handled as a linear continuous effect.

\begin{lstlisting}[language=R, columns=fullflexible, basicstyle=\linespread{0.85}\small\ttfamily, stringstyle=\color{DarkGreen}, keywordstyle=\color{blue}, commentstyle=\color{DarkGreen},]
caldanaNum <- caldana
caldanaNum$time <- as.numeric(as.character(caldanaNum$time))
mod.num <- asca(compounds ~ time*light, data = caldanaNum)
summary(mod.num)
\end{lstlisting}

\subsubsection{APCA}\label{apca}

ANOVA-PCA (APCA) differs from ASCA by adding the error term to the model
before performing PCA instead of backprojecting errors afterwards.

\begin{lstlisting}[language=R, columns=fullflexible, basicstyle=\linespread{0.85}\small\ttfamily, stringstyle=\color{DarkGreen}, keywordstyle=\color{blue}, commentstyle=\color{DarkGreen},]
# Fit APCA model
modp <- apca(assessment ~ candy*assessor, data=candies)
summary(mod)
\end{lstlisting}

\begin{lstlisting}[language=R, columns=fullflexible, basicstyle=\linespread{0.85}\small\ttfamily, stringstyle=\color{DarkGreen}, keywordstyle=\color{blue}, commentstyle=\color{DarkGreen},]
par.old <- par(mfrow=c(2,1), mar=c(4,4,2,1))
loadingplot(modp, scatter=TRUE, labels="names", main="Candy loadings")
scoreplot(modp, main="Candy scores")
par(par.old)
\end{lstlisting}

\subsubsection{PC-ANOVA}\label{pc-anova}

In PC-ANOVA, a PCA is first applied to the data before the scores are
subjected to ANOVA, effectively reversing the roles in ASCA. This means
there will be one or more ANOVA tables in the summary of the output. In
this example, we have chosen to use the number of components that
explain at least 90\% of the variation of the data.
\begin{lstlisting}[language=R, columns=fullflexible, basicstyle=\linespread{0.85}\small\ttfamily, stringstyle=\color{DarkGreen}, keywordstyle=\color{blue}, commentstyle=\color{DarkGreen},]
mod <- pcanova(assessment ~ candy * assessor, data = candies, ncomp = 0.9)
print(mod)
summary(mod)
\end{lstlisting}

When creating score and loading plots for PC-ANOVA, the `global' scores
and loadings will be shown, but the factors can still be used for
colouring the symbols.
\begin{lstlisting}[language=R, columns=fullflexible, basicstyle=\linespread{0.85}\small\ttfamily, stringstyle=\color{DarkGreen}, keywordstyle=\color{blue}, commentstyle=\color{DarkGreen},]
par.old <- par(mfrow=c(2,1), mar=c(4,4,2,1))
loadingplot(mod, scatter=TRUE, labels="names", main="Global loadings")
scoreplot(mod, main="Global scores")
par(par.old)
\end{lstlisting}

\subsubsection{LiMM-PCA}\label{limm-pca}

