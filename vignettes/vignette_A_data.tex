% Options for packages loaded elsewhere
\PassOptionsToPackage{unicode}{hyperref}
\PassOptionsToPackage{hyphens}{url}
%
\documentclass[
]{article}
\usepackage{amsmath,amssymb}
\usepackage{iftex}
\ifPDFTeX
  \usepackage[T1]{fontenc}
  \usepackage[utf8]{inputenc}
  \usepackage{textcomp} % provide euro and other symbols
\else % if luatex or xetex
  \usepackage{unicode-math} % this also loads fontspec
  \defaultfontfeatures{Scale=MatchLowercase}
  \defaultfontfeatures[\rmfamily]{Ligatures=TeX,Scale=1}
\fi
\usepackage{lmodern}
\ifPDFTeX\else
  % xetex/luatex font selection
\fi
% Use upquote if available, for straight quotes in verbatim environments
\IfFileExists{upquote.sty}{\usepackage{upquote}}{}
\IfFileExists{microtype.sty}{% use microtype if available
  \usepackage[]{microtype}
  \UseMicrotypeSet[protrusion]{basicmath} % disable protrusion for tt fonts
}{}
\makeatletter
\@ifundefined{KOMAClassName}{% if non-KOMA class
  \IfFileExists{parskip.sty}{%
    \usepackage{parskip}
  }{% else
    \setlength{\parindent}{0pt}
    \setlength{\parskip}{6pt plus 2pt minus 1pt}}
}{% if KOMA class
  \KOMAoptions{parskip=half}}
\makeatother
\usepackage{xcolor}
\usepackage[margin=1in]{geometry}
\usepackage{color}
\usepackage{fancyvrb}
\newcommand{\VerbBar}{|}
\newcommand{\VERB}{\Verb[commandchars=\\\{\}]}
\DefineVerbatimEnvironment{Highlighting}{Verbatim}{commandchars=\\\{\}}
% Add ',fontsize=\small' for more characters per line
\usepackage{framed}
\definecolor{shadecolor}{RGB}{248,248,248}
\newenvironment{Shaded}{\begin{snugshade}}{\end{snugshade}}
\newcommand{\AlertTok}[1]{\textcolor[rgb]{0.94,0.16,0.16}{#1}}
\newcommand{\AnnotationTok}[1]{\textcolor[rgb]{0.56,0.35,0.01}{\textbf{\textit{#1}}}}
\newcommand{\AttributeTok}[1]{\textcolor[rgb]{0.13,0.29,0.53}{#1}}
\newcommand{\BaseNTok}[1]{\textcolor[rgb]{0.00,0.00,0.81}{#1}}
\newcommand{\BuiltInTok}[1]{#1}
\newcommand{\CharTok}[1]{\textcolor[rgb]{0.31,0.60,0.02}{#1}}
\newcommand{\CommentTok}[1]{\textcolor[rgb]{0.56,0.35,0.01}{\textit{#1}}}
\newcommand{\CommentVarTok}[1]{\textcolor[rgb]{0.56,0.35,0.01}{\textbf{\textit{#1}}}}
\newcommand{\ConstantTok}[1]{\textcolor[rgb]{0.56,0.35,0.01}{#1}}
\newcommand{\ControlFlowTok}[1]{\textcolor[rgb]{0.13,0.29,0.53}{\textbf{#1}}}
\newcommand{\DataTypeTok}[1]{\textcolor[rgb]{0.13,0.29,0.53}{#1}}
\newcommand{\DecValTok}[1]{\textcolor[rgb]{0.00,0.00,0.81}{#1}}
\newcommand{\DocumentationTok}[1]{\textcolor[rgb]{0.56,0.35,0.01}{\textbf{\textit{#1}}}}
\newcommand{\ErrorTok}[1]{\textcolor[rgb]{0.64,0.00,0.00}{\textbf{#1}}}
\newcommand{\ExtensionTok}[1]{#1}
\newcommand{\FloatTok}[1]{\textcolor[rgb]{0.00,0.00,0.81}{#1}}
\newcommand{\FunctionTok}[1]{\textcolor[rgb]{0.13,0.29,0.53}{\textbf{#1}}}
\newcommand{\ImportTok}[1]{#1}
\newcommand{\InformationTok}[1]{\textcolor[rgb]{0.56,0.35,0.01}{\textbf{\textit{#1}}}}
\newcommand{\KeywordTok}[1]{\textcolor[rgb]{0.13,0.29,0.53}{\textbf{#1}}}
\newcommand{\NormalTok}[1]{#1}
\newcommand{\OperatorTok}[1]{\textcolor[rgb]{0.81,0.36,0.00}{\textbf{#1}}}
\newcommand{\OtherTok}[1]{\textcolor[rgb]{0.56,0.35,0.01}{#1}}
\newcommand{\PreprocessorTok}[1]{\textcolor[rgb]{0.56,0.35,0.01}{\textit{#1}}}
\newcommand{\RegionMarkerTok}[1]{#1}
\newcommand{\SpecialCharTok}[1]{\textcolor[rgb]{0.81,0.36,0.00}{\textbf{#1}}}
\newcommand{\SpecialStringTok}[1]{\textcolor[rgb]{0.31,0.60,0.02}{#1}}
\newcommand{\StringTok}[1]{\textcolor[rgb]{0.31,0.60,0.02}{#1}}
\newcommand{\VariableTok}[1]{\textcolor[rgb]{0.00,0.00,0.00}{#1}}
\newcommand{\VerbatimStringTok}[1]{\textcolor[rgb]{0.31,0.60,0.02}{#1}}
\newcommand{\WarningTok}[1]{\textcolor[rgb]{0.56,0.35,0.01}{\textbf{\textit{#1}}}}
\usepackage{graphicx}
\makeatletter
\def\maxwidth{\ifdim\Gin@nat@width>\linewidth\linewidth\else\Gin@nat@width\fi}
\def\maxheight{\ifdim\Gin@nat@height>\textheight\textheight\else\Gin@nat@height\fi}
\makeatother
% Scale images if necessary, so that they will not overflow the page
% margins by default, and it is still possible to overwrite the defaults
% using explicit options in \includegraphics[width, height, ...]{}
\setkeys{Gin}{width=\maxwidth,height=\maxheight,keepaspectratio}
% Set default figure placement to htbp
\makeatletter
\def\fps@figure{htbp}
\makeatother
\setlength{\emergencystretch}{3em} % prevent overfull lines
\providecommand{\tightlist}{%
  \setlength{\itemsep}{0pt}\setlength{\parskip}{0pt}}
\setcounter{secnumdepth}{-\maxdimen} % remove section numbering
\ifLuaTeX
  \usepackage{selnolig}  % disable illegal ligatures
\fi
\IfFileExists{bookmark.sty}{\usepackage{bookmark}}{\usepackage{hyperref}}
\IfFileExists{xurl.sty}{\usepackage{xurl}}{} % add URL line breaks if available
\urlstyle{same}
\hypersetup{
  pdftitle={A. Data handling},
  hidelinks,
  pdfcreator={LaTeX via pandoc}}

\title{A. Data handling}
\author{}
\date{\vspace{-2.5em}}

\begin{document}
\maketitle

\begin{Shaded}
\begin{Highlighting}[]
\CommentTok{\# Start the HDANOVA R package}
\FunctionTok{library}\NormalTok{(HDANOVA)}
\CommentTok{\#\textgreater{} Registered S3 methods overwritten by \textquotesingle{}multiblock\textquotesingle{}:}
\CommentTok{\#\textgreater{}   method             from}
\CommentTok{\#\textgreater{}   print.multiblock   ade4}
\CommentTok{\#\textgreater{}   summary.multiblock ade4}
\CommentTok{\#\textgreater{} Registered S3 methods overwritten by \textquotesingle{}HDANOVA\textquotesingle{}:}
\CommentTok{\#\textgreater{}   method             from      }
\CommentTok{\#\textgreater{}   loadingplot.asca   multiblock}
\CommentTok{\#\textgreater{}   loadings.asca      multiblock}
\CommentTok{\#\textgreater{}   print.asca         multiblock}
\CommentTok{\#\textgreater{}   print.summary.asca multiblock}
\CommentTok{\#\textgreater{}   scoreplot.asca     multiblock}
\CommentTok{\#\textgreater{}   scores.asca        multiblock}
\CommentTok{\#\textgreater{}   summary.asca       multiblock}
\CommentTok{\#\textgreater{} }
\CommentTok{\#\textgreater{} Attaching package: \textquotesingle{}HDANOVA\textquotesingle{}}
\CommentTok{\#\textgreater{} The following object is masked from \textquotesingle{}package:stats\textquotesingle{}:}
\CommentTok{\#\textgreater{} }
\CommentTok{\#\textgreater{}     loadings}
\end{Highlighting}
\end{Shaded}

\hypertarget{read-from-file}{%
\section{Read from file}\label{read-from-file}}

Data are stored in many different file formats. The following three
examples cover two types of CSV-files and generic flat files.

\begin{Shaded}
\begin{Highlighting}[]
\CommentTok{\# Find directory extdata from the multiblock package}
\NormalTok{mbdir }\OtherTok{\textless{}{-}} \FunctionTok{system.file}\NormalTok{(}\StringTok{\textquotesingle{}extdata/\textquotesingle{}}\NormalTok{, }\AttributeTok{package =} \StringTok{"multiblock"}\NormalTok{)}

\CommentTok{\# Comma separated values, row names in first column}
\NormalTok{meta\_data }\OtherTok{\textless{}{-}} \FunctionTok{read.csv}\NormalTok{(}\FunctionTok{paste0}\NormalTok{(mbdir, }\StringTok{"/meta\_data.csv"}\NormalTok{), }\AttributeTok{row.names =} \DecValTok{1}\NormalTok{)}
\CommentTok{\# If working directory matches file location:}
\CommentTok{\# meta\_data \textless{}{-} read.csv(\textquotesingle{}meta\_data.csv\textquotesingle{}, row.names = 1)}
\NormalTok{meta\_data}
\CommentTok{\#\textgreater{}         temperature colour}
\CommentTok{\#\textgreater{} John           38.0   blue}
\CommentTok{\#\textgreater{} Julia          37.0  green}
\CommentTok{\#\textgreater{} James          37.5   blue}
\CommentTok{\#\textgreater{} Jacob          37.6    red}
\CommentTok{\#\textgreater{} Jane           37.2    red}
\CommentTok{\#\textgreater{} Johanna        37.9  green}

\CommentTok{\# Semi{-}colon separated values (locales where the decimal point is comma),}
\CommentTok{\# no row names}
\NormalTok{proteins }\OtherTok{\textless{}{-}} \FunctionTok{read.csv2}\NormalTok{(}\FunctionTok{paste0}\NormalTok{(mbdir, }\StringTok{"/proteins.csv"}\NormalTok{))}
\NormalTok{proteins}
\CommentTok{\#\textgreater{}         prot1       prot2      prot3}
\CommentTok{\#\textgreater{} 1  0.46532048  0.30183300 {-}1.4654414}
\CommentTok{\#\textgreater{} 2 {-}1.79802081 {-}0.22812232 {-}0.4639203}
\CommentTok{\#\textgreater{} 3 {-}1.92962434 {-}0.40513080  0.1767796}
\CommentTok{\#\textgreater{} 4  0.87437138  0.79843798  0.1234731}
\CommentTok{\#\textgreater{} 5 {-}0.62445278 {-}0.07975479 {-}1.1126332}
\CommentTok{\#\textgreater{} 6 {-}0.07493721  1.09576027  1.2656596}

\CommentTok{\# Blank space separated data without labels}
\NormalTok{genes }\OtherTok{\textless{}{-}} \FunctionTok{read.table}\NormalTok{(}\FunctionTok{paste0}\NormalTok{(mbdir, }\StringTok{"/genes.dat"}\NormalTok{))}
\NormalTok{genes}
\CommentTok{\#\textgreater{}            V1         V2         V3}
\CommentTok{\#\textgreater{} 1  0.39033106 {-}0.5720390  1.9147573}
\CommentTok{\#\textgreater{} 2  0.55352785  0.0948703 {-}0.2239755}
\CommentTok{\#\textgreater{} 3  0.09872346 {-}0.1029385  0.9047138}
\CommentTok{\#\textgreater{} 4 {-}0.59213740 {-}0.6027739  0.6177083}
\CommentTok{\#\textgreater{} 5 {-}0.02350148  0.3572809 {-}0.5168416}
\CommentTok{\#\textgreater{} 6  0.76644845  1.2863428  1.8239298}
\end{Highlighting}
\end{Shaded}

\hypertarget{data-pre-processing}{%
\section{Data pre-processing}\label{data-pre-processing}}

Before analysis, various types of pre-processing may be needed. Centring
and standardising/scaling may be considered the most basic. In R, these
operations are performed column-wise by default, leading to autoscaling.
If these operations are performed on the rows, we perform the standard
normal variate (SNV) instead.

\begin{Shaded}
\begin{Highlighting}[]
\CommentTok{\# Column{-}centring}
\NormalTok{genes\_centred }\OtherTok{\textless{}{-}} \FunctionTok{scale}\NormalTok{(genes, }\AttributeTok{scale=}\ConstantTok{FALSE}\NormalTok{)}
\FunctionTok{colMeans}\NormalTok{(genes\_centred) }\CommentTok{\# Check mean values}
\CommentTok{\#\textgreater{}           V1           V2           V3 }
\CommentTok{\#\textgreater{} 1.850372e{-}17 0.000000e+00 7.401487e{-}17}

\CommentTok{\# Autoscaling}
\NormalTok{genes\_scaled }\OtherTok{\textless{}{-}} \FunctionTok{scale}\NormalTok{(genes)}
\FunctionTok{apply}\NormalTok{(genes\_scaled, }\DecValTok{2}\NormalTok{, sd) }\CommentTok{\# Check standard deviations}
\CommentTok{\#\textgreater{} V1 V2 V3 }
\CommentTok{\#\textgreater{}  1  1  1}

\CommentTok{\# SNV (transpose, autoscale, re{-}transpose)}
\NormalTok{genes\_snv }\OtherTok{\textless{}{-}} \FunctionTok{t}\NormalTok{(}\FunctionTok{scale}\NormalTok{(}\FunctionTok{t}\NormalTok{(genes)))}
\FunctionTok{apply}\NormalTok{(genes\_snv, }\DecValTok{1}\NormalTok{, sd) }\CommentTok{\# Check standard deviations}
\CommentTok{\#\textgreater{} [1] 1 1 1 1 1 1}
\end{Highlighting}
\end{Shaded}

\hypertarget{re-coding-categorical-data}{%
\subsection{Re-coding categorical
data}\label{re-coding-categorical-data}}

Most analysis methods require continuous input data. The
\textbf{meta\_data} \textbf{data.frame} contains a character vector (a
factor in older R versions) of categories. This package has a function
\textbf{dummycode} for converting categorical data to various dummy
formats.

\begin{Shaded}
\begin{Highlighting}[]
\CommentTok{\# Default is sum coding}
\FunctionTok{dummycode}\NormalTok{(meta\_data}\SpecialCharTok{$}\NormalTok{colour)}
\CommentTok{\#\textgreater{}   x1 x2}
\CommentTok{\#\textgreater{} 1  1  0}
\CommentTok{\#\textgreater{} 2  0  1}
\CommentTok{\#\textgreater{} 3  1  0}
\CommentTok{\#\textgreater{} 4 {-}1 {-}1}
\CommentTok{\#\textgreater{} 5 {-}1 {-}1}
\CommentTok{\#\textgreater{} 6  0  1}

\CommentTok{\# Treatment coding}
\FunctionTok{dummycode}\NormalTok{(meta\_data}\SpecialCharTok{$}\NormalTok{colour, }\StringTok{"contr.treatment"}\NormalTok{)}
\CommentTok{\#\textgreater{}   xgreen xred}
\CommentTok{\#\textgreater{} 1      0    0}
\CommentTok{\#\textgreater{} 2      1    0}
\CommentTok{\#\textgreater{} 3      0    0}
\CommentTok{\#\textgreater{} 4      0    1}
\CommentTok{\#\textgreater{} 5      0    1}
\CommentTok{\#\textgreater{} 6      1    0}

\CommentTok{\# Full dummy{-}coding (rank deficient)}
\FunctionTok{dummycode}\NormalTok{(meta\_data}\SpecialCharTok{$}\NormalTok{colour, }\AttributeTok{drop =} \ConstantTok{FALSE}\NormalTok{)}
\CommentTok{\#\textgreater{}   xblue xgreen xred}
\CommentTok{\#\textgreater{} 1     1      0    0}
\CommentTok{\#\textgreater{} 2     0      1    0}
\CommentTok{\#\textgreater{} 3     1      0    0}
\CommentTok{\#\textgreater{} 4     0      0    1}
\CommentTok{\#\textgreater{} 5     0      0    1}
\CommentTok{\#\textgreater{} 6     0      1    0}

\CommentTok{\# Replace categorical with dummy{-}coded, use I() to index by common name}
\NormalTok{meta\_data2 }\OtherTok{\textless{}{-}}\NormalTok{ meta\_data}
\NormalTok{meta\_data2}\SpecialCharTok{$}\NormalTok{colour }\OtherTok{\textless{}{-}} \FunctionTok{I}\NormalTok{(}\FunctionTok{dummycode}\NormalTok{(meta\_data}\SpecialCharTok{$}\NormalTok{colour, }\AttributeTok{drop =} \ConstantTok{FALSE}\NormalTok{))}
\NormalTok{meta\_data2}
\CommentTok{\#\textgreater{}         temperature colour.xblue colour.xgreen colour.xred}
\CommentTok{\#\textgreater{} John           38.0            1             0           0}
\CommentTok{\#\textgreater{} Julia          37.0            0             1           0}
\CommentTok{\#\textgreater{} James          37.5            1             0           0}
\CommentTok{\#\textgreater{} Jacob          37.6            0             0           1}
\CommentTok{\#\textgreater{} Jane           37.2            0             0           1}
\CommentTok{\#\textgreater{} Johanna        37.9            0             1           0}
\NormalTok{meta\_data2}\SpecialCharTok{$}\NormalTok{colour}
\CommentTok{\#\textgreater{}   xblue xgreen xred}
\CommentTok{\#\textgreater{} 1     1      0    0}
\CommentTok{\#\textgreater{} 2     0      1    0}
\CommentTok{\#\textgreater{} 3     1      0    0}
\CommentTok{\#\textgreater{} 4     0      0    1}
\CommentTok{\#\textgreater{} 5     0      0    1}
\CommentTok{\#\textgreater{} 6     0      1    0}
\end{Highlighting}
\end{Shaded}

\hypertarget{data-structures-for-analysis-including-blocks}{%
\section{Data structures for analysis including
blocks}\label{data-structures-for-analysis-including-blocks}}

\hypertarget{create-list-of-blocks}{%
\subsection{Create list of blocks}\label{create-list-of-blocks}}

A simple list of blocks can be created using the \textbf{list()}
function. Naming of the blocks can be done directly or after creation.

\begin{Shaded}
\begin{Highlighting}[]
\CommentTok{\# Direct approach}
\NormalTok{blocks1 }\OtherTok{\textless{}{-}} \FunctionTok{list}\NormalTok{(}\AttributeTok{meta =}\NormalTok{ meta\_data2, }\AttributeTok{proteins =}\NormalTok{ proteins, }\AttributeTok{genes =}\NormalTok{ genes)}

\CommentTok{\# Two{-}step approach}
\NormalTok{blocks2 }\OtherTok{\textless{}{-}} \FunctionTok{list}\NormalTok{(meta\_data2, proteins, genes)}
\FunctionTok{names}\NormalTok{(blocks2) }\OtherTok{\textless{}{-}} \FunctionTok{c}\NormalTok{(}\StringTok{\textquotesingle{}meta\textquotesingle{}}\NormalTok{, }\StringTok{\textquotesingle{}proteins\textquotesingle{}}\NormalTok{, }\StringTok{\textquotesingle{}genes\textquotesingle{}}\NormalTok{)}

\CommentTok{\# Same result}
\FunctionTok{identical}\NormalTok{(blocks1, blocks2)}
\CommentTok{\#\textgreater{} [1] TRUE}

\CommentTok{\# Access by name or number}
\NormalTok{blocks1[[}\StringTok{\textquotesingle{}meta\textquotesingle{}}\NormalTok{]]}
\CommentTok{\#\textgreater{}         temperature colour.xblue colour.xgreen colour.xred}
\CommentTok{\#\textgreater{} John           38.0            1             0           0}
\CommentTok{\#\textgreater{} Julia          37.0            0             1           0}
\CommentTok{\#\textgreater{} James          37.5            1             0           0}
\CommentTok{\#\textgreater{} Jacob          37.6            0             0           1}
\CommentTok{\#\textgreater{} Jane           37.2            0             0           1}
\CommentTok{\#\textgreater{} Johanna        37.9            0             1           0}
\NormalTok{blocks2[[}\DecValTok{1}\NormalTok{]]}
\CommentTok{\#\textgreater{}         temperature colour.xblue colour.xgreen colour.xred}
\CommentTok{\#\textgreater{} John           38.0            1             0           0}
\CommentTok{\#\textgreater{} Julia          37.0            0             1           0}
\CommentTok{\#\textgreater{} James          37.5            1             0           0}
\CommentTok{\#\textgreater{} Jacob          37.6            0             0           1}
\CommentTok{\#\textgreater{} Jane           37.2            0             0           1}
\CommentTok{\#\textgreater{} Johanna        37.9            0             1           0}
\end{Highlighting}
\end{Shaded}

\hypertarget{create-data.frame-of-blocks}{%
\subsection{Create data.frame of
blocks}\label{create-data.frame-of-blocks}}

A \textbf{data.frame} is a convenient storage format for data in R and
can handle many types of variables, e.g.~numeric, logical, character,
factor or matrices. The latter is useful for analyses of data with
shared sample mode.

\begin{Shaded}
\begin{Highlighting}[]
\CommentTok{\# Construct block data.frame from list}
\NormalTok{df1 }\OtherTok{\textless{}{-}} \FunctionTok{block.data.frame}\NormalTok{(blocks1)}

\CommentTok{\# Construct block data.frame from data.frame:}
\CommentTok{\# First merge blocks into data.frame}
\NormalTok{my\_data }\OtherTok{\textless{}{-}} \FunctionTok{cbind}\NormalTok{(meta\_data2, proteins, genes)}
\CommentTok{\# Then construct block data.frame using named }
\CommentTok{\# list of indexes}
\NormalTok{df2 }\OtherTok{\textless{}{-}} \FunctionTok{block.data.frame}\NormalTok{(my\_data, }\AttributeTok{block\_inds =} 
        \FunctionTok{list}\NormalTok{(}\AttributeTok{meta =} \DecValTok{1}\SpecialCharTok{:}\DecValTok{2}\NormalTok{, }\AttributeTok{proteins =} \DecValTok{3}\SpecialCharTok{:}\DecValTok{5}\NormalTok{, }\AttributeTok{genes =} \DecValTok{6}\SpecialCharTok{:}\DecValTok{8}\NormalTok{))}

\CommentTok{\# Same result}
\FunctionTok{identical}\NormalTok{(df1,df2)}
\CommentTok{\#\textgreater{} [1] TRUE}

\CommentTok{\# Access by name or number}
\NormalTok{df1[[}\DecValTok{2}\NormalTok{]]}
\CommentTok{\#\textgreater{}               prot1       prot2      prot3}
\CommentTok{\#\textgreater{} John     0.46532048  0.30183300 {-}1.4654414}
\CommentTok{\#\textgreater{} Julia   {-}1.79802081 {-}0.22812232 {-}0.4639203}
\CommentTok{\#\textgreater{} James   {-}1.92962434 {-}0.40513080  0.1767796}
\CommentTok{\#\textgreater{} Jacob    0.87437138  0.79843798  0.1234731}
\CommentTok{\#\textgreater{} Jane    {-}0.62445278 {-}0.07975479 {-}1.1126332}
\CommentTok{\#\textgreater{} Johanna {-}0.07493721  1.09576027  1.2656596}
\NormalTok{df2[[}\StringTok{\textquotesingle{}proteins\textquotesingle{}}\NormalTok{]]}
\CommentTok{\#\textgreater{}               prot1       prot2      prot3}
\CommentTok{\#\textgreater{} John     0.46532048  0.30183300 {-}1.4654414}
\CommentTok{\#\textgreater{} Julia   {-}1.79802081 {-}0.22812232 {-}0.4639203}
\CommentTok{\#\textgreater{} James   {-}1.92962434 {-}0.40513080  0.1767796}
\CommentTok{\#\textgreater{} Jacob    0.87437138  0.79843798  0.1234731}
\CommentTok{\#\textgreater{} Jane    {-}0.62445278 {-}0.07975479 {-}1.1126332}
\CommentTok{\#\textgreater{} Johanna {-}0.07493721  1.09576027  1.2656596}
\NormalTok{df1[}\FunctionTok{c}\NormalTok{(}\DecValTok{1}\NormalTok{,}\DecValTok{3}\NormalTok{)]}
\CommentTok{\#\textgreater{} $meta}
\CommentTok{\#\textgreater{}         temperature colour.xblue colour.xgreen colour.xred}
\CommentTok{\#\textgreater{} John           38.0            1             0           0}
\CommentTok{\#\textgreater{} Julia          37.0            0             1           0}
\CommentTok{\#\textgreater{} James          37.5            1             0           0}
\CommentTok{\#\textgreater{} Jacob          37.6            0             0           1}
\CommentTok{\#\textgreater{} Jane           37.2            0             0           1}
\CommentTok{\#\textgreater{} Johanna        37.9            0             1           0}
\CommentTok{\#\textgreater{} }
\CommentTok{\#\textgreater{} $genes}
\CommentTok{\#\textgreater{}                  V1         V2         V3}
\CommentTok{\#\textgreater{} John     0.39033106 {-}0.5720390  1.9147573}
\CommentTok{\#\textgreater{} Julia    0.55352785  0.0948703 {-}0.2239755}
\CommentTok{\#\textgreater{} James    0.09872346 {-}0.1029385  0.9047138}
\CommentTok{\#\textgreater{} Jacob   {-}0.59213740 {-}0.6027739  0.6177083}
\CommentTok{\#\textgreater{} Jane    {-}0.02350148  0.3572809 {-}0.5168416}
\CommentTok{\#\textgreater{} Johanna  0.76644845  1.2863428  1.8239298}
\NormalTok{df1[}\SpecialCharTok{{-}}\DecValTok{2}\NormalTok{]}
\CommentTok{\#\textgreater{} $meta}
\CommentTok{\#\textgreater{}         temperature colour.xblue colour.xgreen colour.xred}
\CommentTok{\#\textgreater{} John           38.0            1             0           0}
\CommentTok{\#\textgreater{} Julia          37.0            0             1           0}
\CommentTok{\#\textgreater{} James          37.5            1             0           0}
\CommentTok{\#\textgreater{} Jacob          37.6            0             0           1}
\CommentTok{\#\textgreater{} Jane           37.2            0             0           1}
\CommentTok{\#\textgreater{} Johanna        37.9            0             1           0}
\CommentTok{\#\textgreater{} }
\CommentTok{\#\textgreater{} $genes}
\CommentTok{\#\textgreater{}                  V1         V2         V3}
\CommentTok{\#\textgreater{} John     0.39033106 {-}0.5720390  1.9147573}
\CommentTok{\#\textgreater{} Julia    0.55352785  0.0948703 {-}0.2239755}
\CommentTok{\#\textgreater{} James    0.09872346 {-}0.1029385  0.9047138}
\CommentTok{\#\textgreater{} Jacob   {-}0.59213740 {-}0.6027739  0.6177083}
\CommentTok{\#\textgreater{} Jane    {-}0.02350148  0.3572809 {-}0.5168416}
\CommentTok{\#\textgreater{} Johanna  0.76644845  1.2863428  1.8239298}
\NormalTok{df2[}\FunctionTok{c}\NormalTok{(}\StringTok{\textquotesingle{}proteins\textquotesingle{}}\NormalTok{,}\StringTok{\textquotesingle{}genes\textquotesingle{}}\NormalTok{)]}
\CommentTok{\#\textgreater{} $proteins}
\CommentTok{\#\textgreater{}               prot1       prot2      prot3}
\CommentTok{\#\textgreater{} John     0.46532048  0.30183300 {-}1.4654414}
\CommentTok{\#\textgreater{} Julia   {-}1.79802081 {-}0.22812232 {-}0.4639203}
\CommentTok{\#\textgreater{} James   {-}1.92962434 {-}0.40513080  0.1767796}
\CommentTok{\#\textgreater{} Jacob    0.87437138  0.79843798  0.1234731}
\CommentTok{\#\textgreater{} Jane    {-}0.62445278 {-}0.07975479 {-}1.1126332}
\CommentTok{\#\textgreater{} Johanna {-}0.07493721  1.09576027  1.2656596}
\CommentTok{\#\textgreater{} }
\CommentTok{\#\textgreater{} $genes}
\CommentTok{\#\textgreater{}                  V1         V2         V3}
\CommentTok{\#\textgreater{} John     0.39033106 {-}0.5720390  1.9147573}
\CommentTok{\#\textgreater{} Julia    0.55352785  0.0948703 {-}0.2239755}
\CommentTok{\#\textgreater{} James    0.09872346 {-}0.1029385  0.9047138}
\CommentTok{\#\textgreater{} Jacob   {-}0.59213740 {-}0.6027739  0.6177083}
\CommentTok{\#\textgreater{} Jane    {-}0.02350148  0.3572809 {-}0.5168416}
\CommentTok{\#\textgreater{} Johanna  0.76644845  1.2863428  1.8239298}

\CommentTok{\# Use with formula interface (see other vignettes)}
\CommentTok{\# sopls(meta \textasciitilde{} proteins + genes, data = df1)}

\CommentTok{\# Use with single list interface (see other vignettes)}
\CommentTok{\# mfa(df1[c(1,3)], ncomp = 3)}
\end{Highlighting}
\end{Shaded}


\end{document}
